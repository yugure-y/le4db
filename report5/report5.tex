\documentclass[dvipdfmx]{jarticle}
\usepackage{graphicx}
\usepackage{here}
\usepackage{ascmac}
\usepackage{amsmath,amssymb}
\usepackage[margin=20mm]{geometry}
\usepackage{listings,jlisting} %日本語のコメントアウトをする場合jvlisting(もしくはjlisting)が必要
%ここからソースコードの表示に関する設定
\lstset{
  basicstyle={\ttfamily},
  identifierstyle={\small},
  commentstyle={\smallitshape},
  keywordstyle={\small\bfseries},
  ndkeywordstyle={\small},
  stringstyle={\small\ttfamily},
  frame={tb},
  breaklines=true,
  columns=[l]{fullflexible},
  numbers=left,
  xrightmargin=0zw,
  xleftmargin=3zw,
  numberstyle={\scriptsize},
  stepnumber=1,
  numbersep=1zw,
  lineskip=-0.5ex
}
\setcounter{tocdepth}{4}
\pagestyle{empty}
\begin{document}
\title{計算機科学実験及演習4 データベース 課題5}
\author{1029-32-6611 山田裕晃}
\maketitle

\section{概要}
質問および更新を実行するSQL文を作成する。

\section{課題}
\subsection{関係代数の射影および選択に対応するSQL文}
\begin{lstlisting}
eventdb=# SELECT user_id FROM reservation WHERE event_id = 1;
user_id
---------
       1
       3
(2 rows)
\end{lstlisting}
イベントID1に対する予約が存在するユーザーのIDを取得する処理である。
すなわち京都大学11月祭に来場予約をしているユーザーのID一覧を取得している。

\subsection{関係代数の自然結合に対応するSQL文}
\begin{lstlisting}
eventdb=# SELECT user_id, name FROM reservation NATURAL INNER JOIN event_user WHERE event_id = 1;
user_id | name
---------+------
       1 | 山田
       3 | 下田
(2 rows)
\end{lstlisting}
先程の処理に加え、event\_userテーブルとreservationテーブルを自然結合させることで、user\_idに対応する名前を表示させている。

\subsection{UNIONを含むSQL文}
\begin{lstlisting}
eventdb=#SELECT user_id FROM reservation WHERE event_id = 1 UNION SELECT user_id FROM reservation WHERE event_id = 2;
user_id
---------
       1
       3
       4
       2
(4 rows)
\end{lstlisting}
event\_idが1または2のイベントに予約しているユーザーのID一覧を取得している。

\subsection{EXCEPTを含むSQL文}
\begin{lstlisting}
eventdb=#SELECT user_id FROM reservation WHERE event_id = 2 EXCEPT SELECT user_id FROM reservation WHERE event_id = 1;
user_id
---------
       3
(1 row)
\end{lstlisting}
event\_idが2のイベントに予約しているユーザーのIDから、event\_idが1のイベントに予約しているユーザーのIDを除いたものを表示している。
すなわち、計算機科学実験及演習4に予約し、京都大学11月祭に予約していないユーザーを取得している。

\subsection{DISTINCTを含むSQL文}
\begin{lstlisting}
eventdb=# SELECT DISTINCT name FROM reservation NATURAL INNER JOIN event_user;
 name
------
 伊藤
 山田
 加藤
 下田
(4 rows)
\end{lstlisting}
何らかの予約を行なっているユーザーの名前の一覧を表示させる。DISTICT句を使うことで、名前の重複をさせないようにしている。

\subsection{集合関数(COUNT, SUM, AVG, MAX, MIN)を用いたSQL文}
\begin{lstlisting}
eventdb=# SELECT COUNT(user_id) FROM reservation WHERE event_id = 1;
count
-------
     2
(1 row)
\end{lstlisting}
京都大学11月祭に来場予約をしているユーザーの数を取得している。

\subsection{副質問(sub query)を含むSQL文}
\begin{lstlisting}
eventdb=# SELECT COUNT(user_id) FROM (SELECT user_id FROM reservation WHERE accepted = '1') as T;
 count
-------
     2
(1 row)
\end{lstlisting}
何らかのイベントに対する受付が終わっているユーザーの数を取得している。

\subsection{UPDATEを含むSQL文}
\begin{lstlisting}
eventdb=# UPDATE reservation SET accepted = '1' WHERE token = 'abcde';
UPDATE 1
eventdb=# SELECT * from reservation;
 user_id | event_id | token | accepted
---------+----------+-------+----------
       1 |        2 | fghij | 1
       2 |        2 | klmno | 1
       3 |        1 | pqrst | 0
       4 |        2 | uvwxy | 0
       1 |        1 | abcde | 1
(5 rows)
\end{lstlisting}
tokenが"abcde"の予約の受付状態を1に、すなわち受付済に変更する処理をしている。

\subsection{ORDER BYを含むSQL文}
\begin{lstlisting}
eventdb=# SELECT * FROM reservation ORDER BY user_id, event_id;
 user_id | event_id | token | accepted
---------+----------+-------+----------
       1 |        1 | abcde | 1
       1 |        2 | fghij | 1
       2 |        2 | klmno | 1
       3 |        1 | pqrst | 0
       4 |        2 | uvwxy | 0
(5 rows)
\end{lstlisting}
reservationテーブルをuser\_id, event\_idの順に並べる処理をしている。

\subsection{CREATE VIEWを含むSQL文}
\begin{lstlisting}
eventdb=# CREATE VIEW november_festival AS SELECT user_id, event_id, token, accepted FROM reservation WHERE event_id = '1';
CREATE VIEW
eventdb=# SELECT * FROM november_festival;
 user_id | event_id | token | accepted
---------+----------+-------+----------
       3 |        1 | pqrst | 0
       1 |        1 | abcde | 1
(2 rows)
\end{lstlisting}
event\_idが1、すなわち京都大学11月祭に対する予約のデータのみを表示するviewを定義した。

\end{document}