\documentclass[dvipdfmx]{jarticle}
\usepackage{graphicx}
\usepackage{here}
\usepackage{ascmac}
\usepackage{amsmath,amssymb}
\usepackage[margin=20mm]{geometry}
\usepackage{listings,jlisting} %日本語のコメントアウトをする場合jvlisting(もしくはjlisting)が必要
%ここからソースコードの表示に関する設定
\lstset{
  basicstyle={\ttfamily},
  identifierstyle={\small},
  commentstyle={\smallitshape},
  keywordstyle={\small\bfseries},
  ndkeywordstyle={\small},
  stringstyle={\small\ttfamily},
  frame={tb},
  breaklines=true,
  columns=[l]{fullflexible},
  numbers=left,
  xrightmargin=0zw,
  xleftmargin=3zw,
  numberstyle={\scriptsize},
  stepnumber=1,
  numbersep=1zw,
  lineskip=-0.5ex
}
\setcounter{tocdepth}{4}
\pagestyle{empty}
\begin{document}
\title{計算機科学実験及演習4 データベース 課題5}
\author{1029-32-6611 山田裕晃}
\maketitle

\section{概要}
質問および更新を実行するSQL文を作成する。

\section{課題}
\subsection{関係代数の射影および選択に対応するSQL文}
\begin{lstlisting}
eventdb=# SELECT user_id FROM reservation WHERE event_id = 1;
user_id
---------
       1
       3
(2 rows)
\end{lstlisting}
イベントID1に対する予約が存在するユーザーのIDを取得する処理である。
すなわち京都大学11月祭に来場予約をしているユーザーのID一覧を取得している。

\subsection{関係代数の自然結合に対応するSQL文}
\begin{lstlisting}
eventdb=# SELECT user_id, name FROM reservation NATURAL INNER JOIN event_user WHERE event_id = 1;
user_id | name
---------+------
       1 | 山田
       3 | 下田
(2 rows)
\end{lstlisting}
先程の処理に加え、event\_userテーブルとreservationテーブルを自然結合させることで、user\_idに対応する名前を表示させている。

\subsection{UNIONを含むSQL文}


\subsection{EXCEPTを含むSQL文}

\end{document}