\documentclass[dvipdfmx]{jarticle}
\usepackage{graphicx}
\usepackage{here}
\usepackage{ascmac}
\usepackage{amsmath,amssymb}
\usepackage[margin=20mm]{geometry}
\usepackage{listings,jvlisting} %日本語のコメントアウトをする場合jvlisting(もしくはjlisting)が必要
%ここからソースコードの表示に関する設定
\lstset{
  basicstyle={\ttfamily},
  identifierstyle={\small},
  commentstyle={\smallitshape},
  keywordstyle={\small\bfseries},
  ndkeywordstyle={\small},
  stringstyle={\small\ttfamily},
  frame={tb},
  breaklines=true,
  columns=[l]{fullflexible},
  numbers=left,
  xrightmargin=0zw,
  xleftmargin=3zw,
  numberstyle={\scriptsize},
  stepnumber=1,
  numbersep=1zw,
  lineskip=-0.5ex
}
\setcounter{tocdepth}{4}
\pagestyle{empty}
\begin{document}
\title{計算機科学実験及演習4 データベース 課題3}
\author{1029-32-6611 山田裕晃}
\maketitle

\section{概要}
課題2で求めた従属性集合に基づいて、関係スキーマを再設計する。

\section{課題2での導出事項}
求めた関係スキーマを以下に記す。

自明でない関数従属性は以下の通りである。
\begin{itemize}
  \item {トークン} $\rightarrow$ {ユーザーID}
  \item {トークン} $\rightarrow$ {イベントID}
  \item {トークン} $\rightarrow$ {受付状態}
  \item {ユーザーID} $\rightarrow$ {名前}
  \item {ユーザーID} $\rightarrow$ {パスワード}
  \item {イベントID} $\rightarrow$ {イベント名}
  \item {イベントID} $\rightarrow$ {開催日時}
  \item {イベントID} $\rightarrow$ {開催場所}
  \item {イベントID} $\rightarrow$ {定員}
\end{itemize}

\section{関係スキーマの再設計}

全ての属性を一つにまとめた関係スキーマ「イベント予約」を用意する。
\begin{description}
  \item[イベント予約] {ユーザーID, 名前, パスワード, イベントID, イベント名, 開催日時, 開催場所, 定員, チケット料金, トークン, 受付状態}
\end{description}

この関係スキーマに分解法を適用させていくことでBCNFに正規化する。

{ユーザーID} $\rightarrow$ {名前, パスワード}で分解する。

\begin{description}
  \item[ユーザー] {ユーザーID, 名前, パスワード}
  \item[イベント予約] {ユーザーID, イベントID, イベント名, 開催日時, 開催場所, 定員, チケット料金, トークン, 受付状態}
\end{description}

「ユーザー」はBCNFに正規化された。次に、「イベント予約」を
{イベントID} $\rightarrow$ {イベント名, 開催日時, 開催場所, 定員}で分解する。

\begin{description}
  \item[ユーザー] {ユーザーID, 名前, パスワード}
  \item[イベント] {イベントID, イベント名, 開催日時, 開催場所, 定員}
  \item[予約] {ユーザーID, イベントID, トークン, 受付状態}
\end{description}

以上で関係スキーマが全てBCNFに正規化された。
決定した関係スキーマを再掲する。

\begin{description}
  \item[ユーザー] \underline{ユーザーID}, 名前, パスワード
  \item[イベント] \underline{イベントID}, イベント名, 開催日時, 開催場所, 定員
  \item[予約] \underline{ユーザーID}, \underline{イベントID}, トークン, 受付状態
\end{description}

\end{document}