\documentclass[dvipdfmx]{jarticle}
\usepackage{graphicx}
\usepackage{here}
\usepackage{ascmac}
\usepackage{amsmath,amssymb}
\usepackage[margin=20mm]{geometry}
\usepackage{listings,jvlisting} %日本語のコメントアウトをする場合jvlisting(もしくはjlisting)が必要
%ここからソースコードの表示に関する設定
\lstset{
  basicstyle={\ttfamily},
  identifierstyle={\small},
  commentstyle={\smallitshape},
  keywordstyle={\small\bfseries},
  ndkeywordstyle={\small},
  stringstyle={\small\ttfamily},
  frame={tb},
  breaklines=true,
  columns=[l]{fullflexible},
  numbers=left,
  xrightmargin=0zw,
  xleftmargin=3zw,
  numberstyle={\scriptsize},
  stepnumber=1,
  numbersep=1zw,
  lineskip=-0.5ex
}
\setcounter{tocdepth}{4}
\pagestyle{empty}
\begin{document}
\title{計算機科学実験及演習4 データベース 課題3}
\author{1029-32-6611 山田裕晃}
\maketitle

\section{概要}
課題2で求めた従属性集合に基づいて、関係スキーマを再設計する。

\section{課題2での導出事項}
求めた関係スキーマを以下に記す。

\begin{description}
  \item[予約] \underline{トークン}, ユーザーID, イベントID, 受付状態
  \item[ユーザー] \underline{ユーザーID}, 名前, パスワード
  \item[イベント] \underline{イベントID}, イベント名, 開催日時, 開催場所, 定員
  \item[主催者] \underline{ユーザーID}, \underline{イベントID}  
  \item[スタッフ] \underline{ユーザーID}, \underline{イベントID} 
\end{description}

自明でない関数従属性は以下の通りである。
\begin{itemize}
  \item {トークン} $\rightarrow$ {ユーザーID}
  \item {トークン} $\rightarrow$ {イベントID}
  \item {トークン} $\rightarrow$ {受付状態}
  \item {ユーザーID, イベントID} $\rightarrow$ {トークン}
  \item {ユーザーID, イベントID} $\rightarrow$ {受付状態}
  \item {ユーザーID} $\rightarrow$ {名前}
  \item {ユーザーID} $\rightarrow$ {パスワード}
  \item {イベントID} $\rightarrow$ {イベント名}
  \item {イベントID} $\rightarrow$ {開催日時}
  \item {イベントID} $\rightarrow$ {開催場所}
  \item {イベントID} $\rightarrow$ {定員}
\end{itemize}

以下の自明でないかつ関数従属性でない多値従属性が存在する。
\begin{itemize}
  \item {トークン} $\rightarrow \rightarrow$ {ユーザーID, イベントID}
\end{itemize}

\section{関係スキーマの再設計}

{ユーザーID, イベントID} $\rightarrow$ {トークン}及び{ユーザーID, イベントID} $\rightarrow$ {受付状態}は
関係「予約」のキーを含まないが、{ユーザーID, イベントID}は関係「予約」の候補キーであるため、第3正規形である。

ここで、関係「予約」を{ユーザーID, イベントID} $\rightarrow$ {トークン}で分解することでボイスコッド正規形にすることも考えられるが、
この場合、トークン・ユーザーID・イベントIDはいずれも更新されることがない(生成時に自動的に値が決定されるのみ)。
そのため関係「予約」の更新時不整合については考慮しなくてよいため、関係スキーマはこのままとする。

決定した関係スキーマを再掲する。

\begin{description}
  \item[予約情報] \underline{トークン}, ユーザーID, イベントID, 受付状態
  \item[ユーザー] \underline{ユーザーID}, 名前, パスワード
  \item[イベント] \underline{イベントID}, イベント名, 開催日時, 開催場所, 定員
  \item[主催者] \underline{ユーザーID}, \underline{イベントID}  
  \item[スタッフ] \underline{ユーザーID}, \underline{イベントID} 
\end{description}

\end{document}