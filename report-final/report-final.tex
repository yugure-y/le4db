\section{課題1〜3からの変更事項}
課題3で設計していた関係スキーマでは、関係「イベントスタッフ」が表現できていなかったため、
新たに「スタッフ」を追加することにした。

また、主催者はスタッフの一部とみなし、関係「イベントスタッフ」に「スタッフ権限」という名前の属性を追加することで
主催者か否かの判別をする仕様に変更した。

「スタッフ」は、属性にユーザーIDとイベントIDとスタッフ権限を持ち、主キーは「ユーザーID」と「イベントID」の複合主キーである。

それに伴い、以下の関数従属性が追加される。
\begin{itemize}
  \item {ユーザーID, イベントID} $\rightarrow$ {スタッフ権限}
\end{itemize}

新たに正規化しなおした関係スキーマを提示する。
\begin{description}
  \item[ユーザー] \underline{ユーザーID}, 名前, パスワード
  \item[イベント] \underline{イベントID}, イベント名, 開催日時, 開催場所, 定員
  \item[予約] ユーザーID, イベントID, \underline{トークン}, 受付状態
  \item[スタッフ] \underline{ユーザーID}, \underline{イベントID}, スタッフ権限
\end{description}