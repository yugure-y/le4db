\documentclass[dvipdfmx]{jarticle}
\usepackage{graphicx}
\usepackage{here}
\usepackage{ascmac}
\usepackage{amsmath,amssymb}
\usepackage[margin=20mm]{geometry}
\usepackage{listings,jlisting} %日本語のコメントアウトをする場合jvlisting(もしくはjlisting)が必要
%ここからソースコードの表示に関する設定
\lstset{
  basicstyle={\ttfamily},
  identifierstyle={\small},
  commentstyle={\smallitshape},
  keywordstyle={\small\bfseries},
  ndkeywordstyle={\small},
  stringstyle={\small\ttfamily},
  frame={tb},
  breaklines=true,
  columns=[l]{fullflexible},
  numbers=left,
  xrightmargin=0zw,
  xleftmargin=3zw,
  numberstyle={\scriptsize},
  stepnumber=1,
  numbersep=1zw,
  lineskip=-0.5ex
}
\setcounter{tocdepth}{4}
\pagestyle{empty}
\begin{document}
\title{計算機科学実験及演習4 データベース 課題4}
\author{1029-32-6611 山田裕晃}
\maketitle

\section{概要}
課題1,2,3で設計したデータベースを実際にPostgreSQLもしくはSQLiteで構築し、SQLによる検索文を作成する。

\section{関数従属性・正規形に関する考察}

表をDBMSで定義する場合は、主キーを設定することによって従属性を保持することができる。
主キーを設定することで、【主キーの属性】→【その他の属性】という関数従属性が満たされるようになる。

これを正規形の定義に当てはめて考えると、関係データベースにおいては、主キーを設定することによって
その関係表において第3正規形を保つことができると言える。

あるテーブルに外部キーを設定し、【このテーブルの主キー】→【外部キー】という従属性を持っているとする。
その外部キーに何らかの属性が関数従属していた場合、このテーブルの主キーとの間に推移従属性が生じる。
推移従属性が存在すると第3正規形の条件を満たさなくなるので、注意が必要である。

\section{関係表の定義}

関係表を定義するためのSQL文は以下の通りである。

\begin{lstlisting}[caption=definition]
CREATE TABLE event_user(
  user_id INT PRIMARY KEY,
  name VARCHAR(32),
  password VARCHAR(32)
);

CREATE TABLE event(
  event_id INT PRIMARY KEY,
  title VARCHAR(32),
  date DATE,
  place VARCHAR(32),
  capacity INT
);

CREATE TABLE reservation(
  user_id INT,
  event_id INT,
  token varchar(32) PRIMARY KEY,
  accepted bit,
);
\end{lstlisting}

ユーザー・イベント・予約のそれぞれの表では、
ユーザーID・イベントID・トークンをそれぞれ主キーに設定している。
よって、関数従属性は保持される。

\section{データの作成・挿入}

下図のようなデータを作成する。

\begin{table}[H]
  \centering
   \begin{tabular}{|r|l|l|}
    \hline
    ユーザーID & 名前 & パスワード \\
    \hline \hline
    1 & 山田 & yamada \\
    2 & 伊藤 & ito \\
    3 & 下田 & shimoda \\
    4 & 加藤 & kato \\
    \hline
  \end{tabular}
  \caption{ユーザー}
\end{table}

\begin{table}[H]
  \centering
   \begin{tabular}{|r|l|r|l|r|}
    \hline
    イベントID & イベント名 & 日時 & 場所 & 定員 \\
    \hline \hline
    1 & 京都大学11月祭 & 2022/11/19 & 京都大学 & 1000 \\
    2 & 計算機科学実験及演習4 & 2022/10/20 & 総合研究7号館 & 40 \\
    \hline
  \end{tabular}
  \caption{イベント}
\end{table}

\begin{table}[H]
  \centering
   \begin{tabular}{|r|r|l|r|}
    \hline
    ユーザーID & イベントID & トークン & 受付状態 \\
    \hline \hline
    1 & 1 & abcde & 0 \\
    1 & 2 & fghij & 1 \\
    2 & 2 & klmno & 1 \\
    3 & 1 & pqrst & 0 \\
    4 & 2 & uvwxy & 0 \\
    \hline
  \end{tabular}
  \caption{予約}
\end{table}

データを挿入するためのSQL文は以下の通りである。

\begin{lstlisting}[caption=insertion]
  INSERT INTO event_user (user_id, name, password) VALUES 
    (1, '山田', 'yamada'),
    (2, '伊藤', 'ito'),
    (3, '下田', 'shimoda'),
    (4, '加藤', 'kato');

  INSERT INTO event (event_id, title, date, place, capacity) VALUES 
    (1, '京都大学11月祭', '2022-11-19', '京都大学', 1000),
    (2, '計算機科学実験及演習4', '2022-10-20', '総合研究7号館', 40);

  INSERT INTO reservation (user_id, event_id, token, accepted) VALUES 
    (1, 1, 'abcde', '0'),
    (1, 2, 'fghij', '1'),
    (2, 2, 'klmno', '1'),
    (3, 1, 'pqrst', '0'),
    (4, 2, 'uvwxy', '0');
\end{lstlisting}

データを挿入した表の出力結果はそれぞれ以下の通りである。

\begin{lstlisting}[caption=insertion\_result]
eventdb=# select * from event_user;
 user_id | name | password
---------+------+----------
       1 | 山田 | yamada
       2 | 伊藤 | ito
       3 | 下田 | shimoda
       4 | 加藤 | kato
(4 rows)

eventdb=# select * from event;
 event_id |         title         |    date    |     place     | capacity
----------+-----------------------+------------+---------------+----------
        1 | 京都大学11月祭        | 2022-11-19 | 京都大学      |     1000
        2 | 計算機科学実験及演習4 | 2022-10-20 | 総合研究7号館 |       40
(2 rows)

eventdb=# select * from reservation;
 user_id | event_id | token | accepted
---------+----------+-------+----------
       1 |        1 | abcde | 0
       1 |        2 | fghij | 1
       2 |        2 | klmno | 1
       3 |        1 | pqrst | 0
       4 |        2 | uvwxy | 0
(5 rows)
\end{lstlisting}

\end{document}